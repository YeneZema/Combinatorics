\RequirePackage[l2tabu, orthodox]{nag}
\documentclass{article}

\usepackage{amsmath,amsfonts,amsthm,amssymb}
\usepackage[margin=1in]{geometry}
\usepackage{graphicx}
\usepackage{microtype}
\usepackage{siunitx}
\usepackage{booktabs}
\usepackage[colorlinks=false, pdfborder={0 0 0}]{hyperref}
\usepackage{cleveref}

% various theorems, numbered by section

\newtheorem{example}{Example}[section]
\newtheorem{thm}{Theorem}[section]
\newtheorem{lem}[thm]{Lemma}
\newtheorem{prop}[thm]{Proposition}
\newtheorem{cor}[thm]{Corollary}
\newtheorem{conj}[thm]{Conjecture}

\theoremstyle{definition}
\newtheorem{defn}[thm]{Definition}
\newtheorem{defns}[thm]{Definitions}
\newtheorem{con}[thm]{Construction}
\newtheorem{exmp}[thm]{Example}
\newtheorem{exmps}[thm]{Examples}
\newtheorem{notn}[thm]{Notation}
\newtheorem{notns}[thm]{Notations}
\newtheorem{addm}[thm]{Addendum}
\newtheorem{exer}[thm]{Exercise}

\theoremstyle{remark}
\newtheorem{rem}[thm]{Remark}
\newtheorem{rems}[thm]{Remarks}
\newtheorem{warn}[thm]{Warning}
\newtheorem{sch}[thm]{Scholium}
\DeclareMathOperator{\id}{id}

\newcommand{\HRule}[1]{\rule{\linewidth}{#1}} 	% Horizontal rule

\makeatletter							% Title
\def\printtitle{%						
    {\centering \@title\par}}
\makeatother									

\makeatletter							% Author
\def\printauthor{%					
    {\centering \large \@author}}				
\makeatother							

% ------------------------------------------------------------------------------
% Metadata (Change this)
% ------------------------------------------------------------------------------
\title{	\normalsize \textsc{} 									% Upper rule
			\LARGE \textbf{\uppercase{Pigeonhole Principle}}	% Title
			\HRule{1pt} \\ [0.5cm]								% Lower rule + 0.5cm spacing
			\normalsize \today									% Todays date
		}

\author{
		Miliyon T.\\	
		Addis Ababa University\\	
		Department of Mathematics\\
        \texttt{miliyon@ymail.com} \\
}

\begin{document}
\maketitle

\section{Introduction}
\begin{defn}\footnote{Peter Gustav Lejeune Dirichlet, ($1805 - 1859$).}
If $n+1$ objects(pigeons) are placed into $n$ boxes (pigeonholes), then at least one box contains more than one object.
\end{defn}

\begin{defn}[\textbf{Generalized pigeonhole principle}]
If $m$ objects are placed into $k$ boxes, then some box contains at least $\lceil\frac{m}{k}\rceil$ objects.
\end{defn}
\section{Application}

\begin{exmp}
Given $m$ integers $a_1,a_2,\ldots,a_m$ there exists $k$ and $l$ with $1\leq k<l\leq m$, such that
\[a_{k+1}+a_{k+2}+\cdots+a_l\]
is divisible by m.
\end{exmp}
\begin{proof}[Solution]
Consider the following sums
\[a_1,a_1+a_2,a_1+a_2+a_3, \ldots , a_1+a_2+\cdots+a_m \]
If one of the above sum is divisible by $m$, then we are done. But if we got a non-zero remainder when we divide the above sums by $m$, then we will get
\[1,2,\ldots , m-1 \text{ as remainder.}\]
Since there are $m$ sums, and $m-1$ possible values for remainder. Therefore there are integer $k$ and $l$ with $k<l$ such that $a_1+a_2+\cdots+a_k$ and $a_1+a_2+\cdots+a_l$ have the same remainder $r$ when divided by $m$. Thus,
\begin{align}
a_1+a_2+\cdots+a_k=am+r \label{pigex1}\\
\qquad a_1+a_2+\cdots+a_l=bm+r\label{pigex2}
\end{align}
Now subtract (\ref{pigex2}) from (\ref{pigex1})
\begin{align*}
\Rightarrow a_{k+1}+a_{k+2}+\cdots+a_l=(b-a)m
\end{align*}
Hence
\begin{align*}
m|(a_{k+1}+a_{k+2}+\cdots+a_l)
\end{align*}
\end{proof}
\end{document}
