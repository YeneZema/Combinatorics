\documentclass[paper=a4, fontsize=12pt]{scrartcl} % A4 paper and 11pt font size

\usepackage[T1]{fontenc} % Use 8-bit encoding that has 256 glyphs
\usepackage{fourier} % Use the Adobe Utopia font for the document - comment this line to return to the LaTeX default
\usepackage[english]{babel} % English language/hyphenation
\usepackage{amsmath,amsfonts,amsthm} % Math packages
\usepackage{graphicx,wrapfig,lipsum}
\usepackage{fancyhdr} % Custom headers and footers
\pagestyle{fancyplain} % Makes all pages in the document conform to the custom headers and footers
\usepackage[margin=1in]{geometry}  % set the margins to 1in on all sides
\usepackage[hidelinks]{hyperref}
\usepackage{cleveref}
\usepackage{pgf,tikz}
\usepackage{mathrsfs}
\usetikzlibrary{arrows}

\newtheorem{thm}{Theorem}[section]
\newtheorem{lem}[thm]{Lemma}
\newtheorem{prop}[thm]{Proposition}
\newtheorem{cor}[thm]{Corollary}
\newtheorem{conj}[thm]{Conjecture}
\usepackage{sectsty} % Allows customizing section commands
\allsectionsfont{\centering \normalfont\scshape} % Make all sections centered, the default font and small caps



\fancyhead{} % No page header - if you want one, create it in the same way as the footers below
\fancyfoot[L]{} % Empty left footer
\fancyfoot[C]{} % Empty center footer
\fancyfoot[R]{\thepage} % Page numbering for right footer
\renewcommand{\headrulewidth}{0pt} % Remove header underlines
\renewcommand{\footrulewidth}{0pt} % Remove footer underlines
\setlength{\headheight}{13.6pt} % Customize the height of the header

% various theorems, numbered by section

\theoremstyle{definition}
\newtheorem{defn}[thm]{Definition}
\newtheorem{defns}[thm]{Definitions}
\newtheorem{con}[thm]{Construction}
\newtheorem{exmp}[thm]{Example}
\newtheorem{exmps}[thm]{Examples}
\newtheorem{notn}[thm]{Notation}
\newtheorem{notns}[thm]{Notations}
\newtheorem{addm}[thm]{Addendum}
\newtheorem{exer}[thm]{Exercise}

\theoremstyle{remark}
\newtheorem{rem}[thm]{Remark}
\newtheorem{rems}[thm]{Remarks}
\newtheorem{warn}[thm]{Warning}
\newtheorem{sch}[thm]{Scholium}
\DeclareMathOperator{\id}{id}

\newcommand{\bd}[1]{\mathbf{#1}}  % for bolding symbols
\newcommand{\RR}{\mathbb{R}}      % for Real numbers
\newcommand{\ZZ}{\mathbb{Z}}      % for Integers
\newcommand{\col}[1]{\left[\begin{matrix} #1 \end{matrix} \right]}
\newcommand{\comb}[2]{\binom{#1^2 + #2^2}{#1+#2}}
%----------------------------------------------------------------------------------------
%	TITLE SECTION
%----------------------------------------------------------------------------------------

\newcommand{\horrule}[1]{\rule{\linewidth}{#1}} % Create horizontal rule command with 1 argument of height

\title{	
\normalfont \normalsize
\textsc{Addis Ababa university} \\ [25pt] % Your university, school and/or department name(s)
\horrule{0.5pt} \\[0.4cm] % Thin top horizontal rule
\huge Conjectures \\ % The assignment title
\horrule{2pt} \\[0.5cm] % Thick bottom horizontal rule
}

\author{Miliyon T.} % Your name

\date{\normalsize\today} % Today's date or a custom date

\begin{document}

\definecolor{ffffff}{rgb}{1.,1.,1.}
\definecolor{qqqqff}{rgb}{0.,0.,1.}
\maketitle % Print the title
\section{Conjecture}

\begin{conj}\footnote{Except for $n-1=12k$, where $k\in \mathbb{N}$.}
$$
\binom{n}{3}=
\begin{cases}
odd,&\mbox{ where } n \mbox{ is prime and } n-2 \mbox{ is its twin }\\
even,&\mbox{ otherwise }
\end{cases}
$$
\end{conj}

\begin{proof}(Intuition)
\begin{align*}
\binom{n}{3}&=\frac{n!}{(n-3)!3!}\\
            &=\frac{n(n-1)(n-2)}{6}
\end{align*}
Now we have 
$$
\binom{n}{3}=
\begin{cases}
even &\mbox{if } 12|n(n-1)(n-2)\\
odd,&\mbox{ otherwise }
\end{cases}
$$
For a trivial reason let $n>3$.\\

\begin{description}
  \item[Case 1:] If $n$ is even, then is $n-2$. For an obvious reason, from $n$, $n-1$ and $n-2$ one of them is a multiple of 3. Hence $12|n(n-1)(n-2)$.
  \item[Case 2:] If $n$ is odd.
\end{description}
\end{proof}

\newpage
\begin{conj}\label{conj2}
The number of possible triangles in a complete graph of order $n$ ($K_n$) is 
$$
\begin{cases}
\frac{n^2(n-2)^2(n-1)^2)}{6^2},&\mbox{ where } \binom{n}{3} \mbox{ is odd.}\\
\frac{(n^2-n)(n^2-4)(n^2-5n+12)}{12^2} ,&\mbox{where }\binom{n}{3} \mbox{is even.}
\end{cases}
$$
\end{conj}

\begin{proof}(Intuition)
\end{proof}
Let's look at some lower cases 

There are 8 possible triangles in $K_4$(Exercise).

\begin{tikzpicture}[line cap=round,line join=round,>=triangle 45,x=1.0cm,y=1.0cm]
\clip(0.2,0.015) rectangle (15.66,5.915);
\draw (4.46,4.495)-- (8.5,1.335);
\draw (8.5,4.495)-- (4.48,1.335);
\draw (4.46,4.495)-- (8.5,4.495);
\draw (4.48,1.335)-- (4.46,4.495);
\draw (8.5,1.335)-- (4.48,1.335);
\draw (8.5,4.495)-- (8.5,1.335);
\begin{scriptsize}
\draw [fill=qqqqff] (4.46,4.495) circle (1.5pt);
\draw [fill=qqqqff] (4.48,1.335) circle (1.5pt);
\draw [fill=qqqqff] (8.5,1.335) circle (1.5pt);
\draw [fill=qqqqff] (8.5,4.495) circle (1.5pt);
\end{scriptsize}
\end{tikzpicture}

Now, let us compute the $\#$ of possible triangle in $K_4$ using (\ref{conj2}). So $n=4$ and $\binom{4}{3}=4$ which is even. Thus, our formula is $\frac{(n^2-n)(n^2-4)(n^2-5n+12)}{12^2}$. Plugging $4$ in place of $n$ gives
$$\frac{(4^2-4)(4^2-4)(4^2-5(4)+12)}{12^2}=8$$

There are 35 possible triangles in $K_5$(Exercise).

\begin{tikzpicture}[line cap=round,line join=round,>=triangle 45,x=1.0cm,y=1.0cm]
\clip(0.32,1.635) rectangle (15.78,8.575);
\fill[color=ffffff,fill=ffffff,fill opacity=0.1] (6.86,8.015) -- (4.28,5.595) -- (5.36,2.395) -- (8.28,2.395) -- (9.46,5.575) -- cycle;
\draw [color=ffffff] (6.86,8.015)-- (4.28,5.595);
\draw [color=ffffff] (4.28,5.595)-- (5.36,2.395);
\draw [color=ffffff] (5.36,2.395)-- (8.28,2.395);
\draw [color=ffffff] (8.28,2.395)-- (9.46,5.575);
\draw [color=ffffff] (9.46,5.575)-- (6.86,8.015);
\draw (6.86,8.015)-- (5.36,2.395);
\draw (4.28,5.595)-- (8.28,2.395);
\draw (9.46,5.575)-- (5.36,2.395);
\draw (6.86,8.015)-- (8.28,2.395);
\draw (4.28,5.595)-- (9.46,5.575);
\draw (6.86,8.015)-- (4.28,5.595);
\draw (5.36,2.395)-- (4.28,5.595);
\draw (8.28,2.395)-- (5.36,2.395);
\draw (9.46,5.575)-- (8.28,2.395);
\draw (6.86,8.015)-- (9.46,5.575);
\begin{scriptsize}
\draw [fill=qqqqff] (6.86,8.015) circle (1.5pt);
\draw [fill=qqqqff] (4.28,5.595) circle (1.5pt);
\draw [fill=qqqqff] (5.36,2.395) circle (1.5pt);
\draw [fill=qqqqff] (8.28,2.395) circle (1.5pt);
\draw [fill=qqqqff] (9.46,5.575) circle (1.5pt);
\end{scriptsize}
\end{tikzpicture}

To compute the $\#$ of possible triangle in $K_5$ set $n=5$. Then $\binom{5}{3}=10$ which is even
\begin{align*}
\frac{(n^2-n)(n^2-4)(n^2-5n+12)}{12^2} &=\frac{(5^2-5)(5^2-4)(5^2-5(5)+12)}{12^2}\\
                                       &=35
\end{align*}

Using (\ref{conj2}) we predict the $\#$ of possible triangle in $K_6$ to be $120$.


\newpage
\begin{thebibliography}{9}

\bibitem{May}
[Boris A. Kordemsky] ~
The Moscaow Puzzles, $359$ Mathematical recreations.


\end{thebibliography}
\end{document}
