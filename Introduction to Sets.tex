\documentclass[11pt]{article}
\usepackage[margin=1in]{geometry}								% Package to create dummy text
\usepackage{amsmath,amsfonts,amsthm,amssymb}					% Math packages
\usepackage[pdftex]{graphicx}									% Enable pdflatex
\usepackage{hyperref} % URLs

\newtheorem{thm}{Theorem}[section]
\newtheorem{lem}[thm]{Lemma}
\newtheorem{prop}[thm]{Proposition}
\newtheorem{cor}[thm]{Corollary}
\newtheorem{conj}[thm]{Conjecture}

\theoremstyle{definition}
\newtheorem{defn}[thm]{Definition}
\newtheorem{defns}[thm]{Definitions}
\newtheorem{con}[thm]{Construction}
\newtheorem{exmp}[thm]{Example}
\newtheorem{notn}[thm]{Notation}
\newtheorem{notns}[thm]{Notations}
\newtheorem{addm}[thm]{Addendum}
\newtheorem{exer}[thm]{Exercise}
\newtheorem{rem}[thm]{Remark}
\theoremstyle{plain}



\begin{document}
\nocite{}

\title{Sets}

\author{Miliyon T.}
\date{October 7, 2013}
\maketitle

\section{Introduction to Sets}

Sets are collections of objects. Here are three examples:

1) The set of all snakes in the world;
2) The set of all leaves in a given forest;
3) The set of all two-digit integers that are divisible by \( 3 \);

In some cases it is possible to list all elements of the set. One such example is the set \( S \) of all two-digit integers that are divisible by \( 17 \). We can write \[ S=\{17,34,51,68,85\}.\] The order in which we list the elements does not matter. This means that \( \{17,34,51,68,85\} \) is precisely the same set as \( \{34,51,17,68,85\} \). Also, repetitions of elements do not affect sets. The set \( \{17 \), \( 17 \), \( 34 \), \( 34 \), \( 51 \), \( 68 \), \( 85\} \) is the same old \( S \).

Usually we denote sets by capital letters from the alphabet.

\subsection{Elements and Subsets}

If we want to emphasize that something is an element of a set we use the symbol \( \in \). In the case of previously defined set \( S \) we write \( 34\in S \). We also use \( \not\in \) to emphasize that something is not in the set. This is how we do that: \( 23\not\in S \).

A set \( Q \) is a subset of a set \( S \) if every element of \( Q \) is also an element of \( S \). The sentence
''\( Q \) is a subset of \( S \)''

is written as \( Q\subseteq S \). Example:
\[ \{34,68\}\subseteq S.\]
Union, intersection, and difference
The union \( A\cup B \) of two sets \( A \) and \( B \) is a set containing all elements that are in either \( A \) or \( B \). The intersection \( A\cap B \) is the set containing elements that are in both \( A \) and \( B \).

Let \( S=\{17,34,51,68,85\} \) and \( T=\{5, 7, 17, 71, 85\} \) be two sets. Then
\[ S\cup T=\{5,7,17,34,51,71,58,85\}\;\;\mbox{and}\;\;\; S\cap T=\{17,85\}.\]
The difference \( A\setminus B \) is the set containing those elements of \( A \) that are not in \( B \). In the case of previously defined \( S \) and \( T \) we have \[ S\setminus T=\{34,51,68\}.\]
Complement of a set
If we know that all elements we are dealing with are elements of a given fixed set \( U \), then we have a special notion for \( U\setminus S \). We write \[ S^C=U\setminus S\] and say \( S^C \) is a complement of \( S \).

Then it is easy to prove the following identities: \[ (A\cup B)^C=A^C\cap B^C;\] \[ (A\cap B)^C=A^C\cup B^C;\] \[ A\setminus B=A\cap B^C.\] The above identities are called de Morgan's laws.
Other ways to write sets
So far we described sets by listing all of their elements. We want to study those sets where this cannot be done. We studied the set \( S \) of all two-digit positive integers that are divisible by 17. Consider now the set \( D \) of all positive integers that are divisible by \( 17 \). We can't list all elements of \( D \). However, we can write math propositions involving this set \( D \). Examples are: \[ 85\in D\;\;\; 170\in D\;\;\; 710\not\in D\;\;\; D\cap S=S.\] Although we don't have the list of all elements of \( D \), we are able to guarantee that all the above propositions are correct. If there is a property that uniquely defines the set we use the following syntax involving colon \( : \) to define our set \( D \) \[ D=\{z: \;z\mbox{ is a positive integer that is divisible by }17\}.\]

Some books use vertical line \( | \) instead of colon, but you don't want to do that, because in problem solving vertical line is so precious as it is used for divisibility (we write \( 17|85 \)). In some cultures, the symbol : (colon) is used for division. That should be stopped - using fractions instead is more professional.
Cartesian product of sets

The cartesian product \( A\times B \) of sets \( A \) and \( B \) is the set of ordered pairs whose first component is from \( A \) and the second component is from \( B \). For example, if \( A=\{1,2,3,4\} \) and \( B=\{3, a, 9\} \), then their cartesian product is: \[ A\times B=\{(1,3), (1, a), (1,9), (2,3), (2,a), (2,9), (3,3), (3,a), (3,9), (4,3), (4,a), (4,9)\}.\] We use the notation \( A^2 \) for \( A\times A \).

Cartesian product is used to formally define many important concepts in mathematics. For the beginning, you may notice that we can define the two-dimensional plane as the product of the line \( \mathbb R \) with itself. Later on we will see that relations, functions, and graphs can be defined using cartesian product.

\end{document}